\problemname{Well That's Just Grate}

A grating is defined as ``any regularly spaced collection of essentially identical, parallel, elongated
elements''. Most of the time, this means a single set of parallel elongated elements, but they sometimes also
consist of two sets, with the second set perpendicular to the first. In the grating industry, these are
known as ``grid'' or ``mesh'' gratings, due to the grid shape they naturally form.

George is the Chief Innovator at George and Gary's Great Grating Group, and he's looking to come up with the 
next big idea to disrupt the grating industry. One day, he has his big ``Eureka!'' moment: why must the
elongated elements in gratings be parallel? What if a grating just consisted of elongated metal wires welded
together in arbitrary orientations? Sure, they could be perpendicular or parallel, but the point was they
didn't HAVE to be. Now, all ideas still must have limitations. The grating would still be bounded by an
$L$ by $W$ rectangular border. Then $N$ lines of metal wire would be welded to the grating, with each
being defined as a line segment where both endpoints lie somewhere on the rectangular border.

Most of the gratings manufactured by George and Gary's Great Grating Group are used as storm drains, and
one of the biggest issues when it comes to storm drains is people dropping phones down them. Thus,
given a certain grating design, George wants to know if a phone dropped flat at a particular spot on the
grating will fall through. For this he consults Gary, the Chief Physicist, who tells him the following:

\begin{quotation}
\textit{
    Given that a rectangular phone falls flat on the grate at a certain position, find the points
    at which the edges of the phone intersect metal wire on the grate. Then, form a maximal polygon with
    the points of intersection along the edges of the phone. If the center of gravity of the phone falls
    within this polygon, then the phone will fall into the grate. Otherwise, it will not.
}
\end{quotation}

Knowing this, George has all the tools he needs to determine if dropping the phone in certain positions will
result in the phone falling through the grate. Can you help him in this endeavor?

George assumes that weight is distributed equally across the phone, so its center of gravity will be at the
center of the rectangle.

\section*{Input}

The first line of input consists of four space-separated integers, $L$, $W$, $N$, and $P$: the length of
the grating, the width of the grating, the number of metal wires welded onto the grating, and the
number of phone positions that George is considering. On the coordinate grid, the border of the grating
will be the grid-aligned rectangle with a corner at $(0, 0)$ and the opposite corner at $(L, W)$.

$N$ lines follow, each consisting of four space-separated integers $x_1$, $y_1$, $x_2$, $y_2$, giving the
position of one of the metal wires on the grating. The metal wire is represented as a line segment with
endpoints $(x_1, y_1)$ and $(x_2, y_2)$. $(x_1, y_1)$ and $(x_2, y_2)$ are guaranteed to lie on two
different edges of the rectangular grating.

Finally, $P$ lines follow, each consisting of four space-separated $a_1$, $b_1$, $a_2$, $b_2$, giving the
position at which a phone is dropped. The phone is represented by a grid-aligned rectangle with lower left
corner at $(a_1, b_1)$ and upper right corner at $(a_2, b_2)$. It is guaranteed that $a_2 > a_1$ and
$b_2 > b_1$.

\section*{Output}

For each of the phone drop locations in question, output "Yes" if the phone falls through the grating,
and "No" otherwise, each on a new line of output.
