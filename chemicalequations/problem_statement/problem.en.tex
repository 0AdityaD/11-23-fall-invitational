\problemname{Candy Consumption}

Bob is a master chemist and has a list of $N$ different chemical equations that detail how to create specific
compounds. Each chemical equation consists of a compound and $R$ reactants with associated quantities that can 
be used to make the compound. A base compound is defined as a reactant that cannot be further reduced into other 
reactants according to the list of chemical equations. Given this list of chemical equations and a compound that
is in the list of chemical equations (possibly only as a reactant), output an alphabetized list of base compounds 
and quantities that can be used to form the given compound. Note that the chemical equation that forms the given 
compound will not necessarily contain only base compounds. You are guaranteed that the list of chemical equations
will not form a cycle, meaning  that there exists no compound that can be used to make its constituent reactants.

\section*{Input}
The first line contains a single integer $N$ such that $1 \leq N \leq 1000$.
Next will follow $N$ chemical equations. 

The start of each chemical equation consists of a line that has the name of the compound followed by an integer,
$R$ such that $1 \leq R \leq 10$, that indicates $R$ reactants are used to make the compound. The next $R$ 
lines of the chemical equation consist of a number $q_i$ (the quantity of the component reactant needed to make
the compound) such that $1 \leq q_i \leq 10$ and the component reactant name.

The final line of the input will contain a compound name that was previously in the list of chemical
equations. You will use this compound name and output all the base compounds that are needed to form 
this compound.

\section*{Output}
Given that you can form the desired compound with $B$ base compounds, output $B$ lines (sorted alphabetically)
with each base compound name and the quantity of that base compound needed.
